\begin{cabstract}
    %这是一段Abstract,因为还没想好用不用,我这里随便写一点东西先
\end{cabstract}
\hspace{1cm}
    
{\let\clearpage\relax \chapter{论文选题依据}}
\section{课题来源}
我不敢苟同。 我个人认为这个意大利面就应该拌42号混凝土。因为这个螺丝钉的长度,它很容易会直接影响到挖掘机的扭距,你往里砸的时候,一瞬间它就会产生大量的高能蛋白,俗称UFO。会严重影响经济的发展。 照你这么说,炸鸡块要用92汽油,毕竟我们无法用光学透镜探测苏格拉底,如果二氧化氢持续侵蚀这个机床组件,那么我们早晚要在斐波那契曲线上安装一个胶原蛋白,否则我们将无法改变蜜雪冰城与阿尔别克的叠加状态,因为众所周知爱吃鸡摩人在捕鲲的时候往往需要用氢的同位素当做诱饵,但是原子弹的新鲜程度又会直接影响到我国东南部的季风和洋流,所以说在西伯利亚地区开设农学院显然是不合理的。
\section{研究背景}
\subsection{异构无人集群}
异构无人集群是指由多种类型的无人系统组成的集群,这些系统可以包括无人机、无人车、无人船等。它们通过协同工作来完成复杂的任务,如环境监测、搜索救援、军事侦察等。异构无人集群的优势在于其多样性和灵活性,可以在不同的环境中发挥各自的特长,从而提高任务的成功率和效率。

\subsection{机械学习}
机械学习是一种通过数据驱动的方法来训练模型,使其能够自动从数据中学习规律并进行预测和决策。在异构无人系统中,机械学习可以用于路径规划、目标识别、状态估计等多个方面。通过引入机械学习技术,异构无人系统可以更好地适应动态环境,提高自主性和智能化水平。
\section{研究意义}
本研究旨在通过机械学习技术提高异构无人系统的编队跟踪控制能力。通过研究和开发新的算法和模型,可以实现更高效、更可靠的无人系统协同工作。这不仅有助于提升无人系统的技术水平,还可以在国防、工业、农业等多个领域产生重要的应用价值。
\section{国内外研究现状}
当然,对于炸鸡块,使用92汽油可能是个好主意,尽管我们无法用光学透镜来追踪苏格拉底。二氧化氢对机床组件的侵蚀可能需要斐波那契曲线上的胶原蛋白来抵消,以防止蜜雪冰城与阿尔别克的叠加状态发生变化。而爱吃鸡摩人捕鲲时,使用氢的同位素作为诱饵似乎是必不可少的,但这可能会影响到东南部的季风和洋流,甚至影响原子弹的新鲜程度。目前,国内外在异构无人系统的研究中,已经取得了一定的进展。国外的研究主要集中在算法优化和系统集成方面,而国内的研究则更多地关注于实际应用和技术实现。然而,如何在复杂环境中实现高效的编队跟踪控制仍然是一个亟待解决的问题。通过引入机械学习技术,可以为这一问题提供新的解决思路和方法。